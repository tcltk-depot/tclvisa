\documentclass[12pt, a4paper]{book}
\usepackage[utf8]{inputenc}
\usepackage{hyperref}

\makeindex

\newcommand{\tclvisa}{{\tt tclvisa }}
\newcommand{\indexentry}[2]{\item #1 #2}
\newcommand{\VISA}{\mbox{VISA }}

\newcommand{\COMMANDREF}[1]{{\tt \hyperref[#1]{#1}}}
\newcommand{\VISACOMMANDREF}[1]{{\tt \mbox{#1}}\index{#1}}
\newcommand{\TCLCOMMANDREF}[1]{{\tt \mbox{#1}}\index{#1}}

\newcommand{\SEEALSO}{\subsection{See Also}}
\newcommand{\EXAMPLE}{\subsection{Example}}
\newcommand{\PURPOSE}{\subsection{Purpose}}
\newcommand{\SYNTAX}[1]{\subsection{Syntax}{\tt #1}}
\newcommand{\NOTES}{\subsection{Notes}}
\newcommand{\ARGUMENTS}{\subsubsection{Arguments}}
\newcommand{\RETURN}{\subsubsection{Return Value}}
\newcommand{\NORETURN}{\subsubsection{Return Value} None}
\newcommand{\COMMAND}[1]{\section{#1}\label{#1}}
\newcommand{\BEGINARGUMENTS}{\ARGUMENTS\begin{itemize}}
\newcommand{\ENDARGUMENTS}{\end{itemize}}
\newcommand{\NOARGUMENTS}{\ARGUMENTS None}
\newcommand{\ARGUMENT}[1]{\item {\tt \mbox{#1}}~---}

\newcommand{\ARGCHANNEL}{\ARGUMENT{session} channel containing reference to a \VISA resource session opened by \COMMANDREF{visa::open}.}
\newcommand{\ARGRM}{\ARGUMENT{RMsession} channel containing reference to open Resource Manager session opened by \COMMANDREF{visa::open-default-rm}.}

\title{\tclvisa Programmer Reference Manual}

\begin{document}

\maketitle

\tableofcontents

\chapter{Correspondence Between \VISA Functions and Tcl Commands}

\begin{tabular}{ll}
\VISACOMMANDREF{viClear} & \COMMANDREF{visa::clear}	\\
\VISACOMMANDREF{viClose} & \TCLCOMMANDREF{close}	\\
\VISACOMMANDREF{viFindNext}, \VISACOMMANDREF{viFindRsrc} & \COMMANDREF{visa::find}	\\
\VISACOMMANDREF{viGetAttribute} & \COMMANDREF{visa::get-attribute}	\\
\VISACOMMANDREF{viLock} & \COMMANDREF{visa::lock}	\\
\VISACOMMANDREF{viOpen} & \COMMANDREF{visa::open}	\\
\VISACOMMANDREF{viOpenDefaultRM} & \COMMANDREF{visa::open-default-rm}	\\
\VISACOMMANDREF{viPrintf} & \TCLCOMMANDREF{format}, \TCLCOMMANDREF{puts}	\\
\VISACOMMANDREF{viQueryf} & \TCLCOMMANDREF{format}, \TCLCOMMANDREF{puts}, \TCLCOMMANDREF{gets}, \TCLCOMMANDREF{scan}	\\
\VISACOMMANDREF{viRead} & \TCLCOMMANDREF{read}	\\
\VISACOMMANDREF{viScanf} & \TCLCOMMANDREF{gets}, \TCLCOMMANDREF{scan}	\\
\VISACOMMANDREF{viSetAttribute} & \COMMANDREF{visa::set-attribute}	\\
\VISACOMMANDREF{viUnlock} & \COMMANDREF{visa::unlock}	\\
\VISACOMMANDREF{viWrite} & \TCLCOMMANDREF{puts}	\\
\end{tabular}

\chapter{\tclvisa Command Reference}

%%%%%%%%%%%%%%%%%%%%%%%%%%%%%%%%%%%%%%%%%%%%%%%%%%%%%%%%%%%%%%%%%%%%%

\COMMAND{visa::clear}

\PURPOSE

Clears a device. This command is a front-end for \VISACOMMANDREF{viClear} \VISA API function.

\SYNTAX{visa::clear session}

\BEGINARGUMENTS
\ARGCHANNEL
\ENDARGUMENTS

\NORETURN

\EXAMPLE

\begin{verbatim} 
# open instrument with default access mode and timeout
set vi [visa::open $rm "ASRL1::INSTR"]

# set device to known state
visa::clear $vi
\end{verbatim} 

\SEEALSO

\COMMANDREF{visa::open}

%%%%%%%%%%%%%%%%%%%%%%%%%%%%%%%%%%%%%%%%%%%%%%%%%%%%%%%%%%%%%%%%%%%%%

\COMMAND{visa::find}

\PURPOSE

Queries a VISA system to locate the resources associated with a specified interface. This command is a front-end for \VISACOMMANDREF{viFindRsrc} and \VISACOMMANDREF{viFindNext} \VISA API functions.

\SYNTAX{visa::open RMsession expr}

\BEGINARGUMENTS
\ARGRM
\ARGUMENT{expr} regular expression followed by an optional logical expression. Refer to \VISA API documentation for detailed discussion.
\ENDARGUMENTS

\RETURN

Tcl list with addresses of all resources found. If no resources found that match the given expression, empty list is returned.

\EXAMPLE

\begin{verbatim} 
# open resource manager session
set rm [visa::open-default-rm]

# get addresses of all serial instruments
foreach addr [visa::find $rm "ASRL?*INSTR"] {
  # address is in $addr variable
}
\end{verbatim} 

\SEEALSO

\COMMANDREF{visa::open-default-rm}

%%%%%%%%%%%%%%%%%%%%%%%%%%%%%%%%%%%%%%%%%%%%%%%%%%%%%%%%%%%%%%%%%%%%%

\COMMAND{visa::get-attribute}

\PURPOSE

Retrieves the state of an attribute. This command is a front-end for \VISACOMMANDREF{viGetAttribute} \VISA API function.

\SYNTAX{visa::get-attribute session attribute}

\BEGINARGUMENTS
\ARGCHANNEL
\ARGUMENT{attribute} Integer value with ID of the \VISA attribute to retrieve. Use one of the predefned {\tt visa::ATTR\_XXX} constants.
\ENDARGUMENTS

\RETURN

Attribute value.

\EXAMPLE

\begin{verbatim} 
# open instrument with default access mode and timeout
set vi [visa::open $rm "ASRL1::INSTR"]

# retrieve current baud rate of a serial bus
set baud [visa::get-attribute $vi $visa::ATTR_ASRL_BAUD]
\end{verbatim} 

\SEEALSO

\COMMANDREF{visa::set-attribute}

%%%%%%%%%%%%%%%%%%%%%%%%%%%%%%%%%%%%%%%%%%%%%%%%%%%%%%%%%%%%%%%%%%%%%

\COMMAND{visa::lock}

\PURPOSE

Establishes an access mode to the specified resource. This command is a front-end for \VISACOMMANDREF{viLock} \VISA API function.

\SYNTAX{visa::lock session ?lockType? ?timeout? ?requestedKey?}

\BEGINARGUMENTS
\ARGCHANNEL
\ARGUMENT{lockType} integer value determining type of locking. May be either {\tt visa::EXCLUSIVE\_LOCK} or {\tt visa::SHARED\_LOCK}. If argument is omitted, {\tt visa::EXCLUSIVE\_LOCK} is assumed.
\ARGUMENT{timeout} timeout of getting lock. If argument is omitted, infinite timeout is assumed.
\ARGUMENT{requestedKey} name of the shared lock to acquire. If exclusive locking is requested, this argument is ignored.
\ENDARGUMENTS

\RETURN

\begin{itemize}
\item If an exclusive lock is requiested and acquired, procedure returns nothing.
\item If an shared lock is requiested and acquired, procedure returns name of the lock.
\end{itemize}

\EXAMPLE

\begin{verbatim} 
# get exclusive lock and wait forever
visa::lock $vi

# get exclusive lock and wait 5 seconds
visa::lock $vi $visa::EXCLUSIVE_LOCK 5000

# get shared lock and wait 5 seconds
set key [visa::lock $vi $visa::SHARED_LOCK 5000 "MYLOCK"]
\end{verbatim} 

\SEEALSO

\COMMANDREF{visa::open}, \COMMANDREF{visa::unlock}

%%%%%%%%%%%%%%%%%%%%%%%%%%%%%%%%%%%%%%%%%%%%%%%%%%%%%%%%%%%%%%%%%%%%%

\COMMAND{visa::open}

\PURPOSE

Opens a session to the specified resource. This command is a front-end for \VISACOMMANDREF{viOpen} \VISA API function.

\SYNTAX{visa::open RMsession rsrcName ?accessMode? ?openTimeout?}

\BEGINARGUMENTS
\ARGRM
\ARGUMENT{rsrcName} name of the VISA resource to connect to.
\ARGUMENT{accessMode} integer parameter determining access mode. May be bitwise OR combination of the following constants:
	\begin{itemize}
	\item {\tt visa::EXCLUSIVE\_LOCK}~--- acquire exclusive lock to a resource;
	\item {\tt visa::LOAD\_CONFIG}~--- use external configuration;
	\end{itemize}
	Refer to \VISA documentation for more details about access mode. If parameter is omitted, default zero value is used.
\ARGUMENT{openTimeout} operation timeout. If parameter is omitted, default timeout value is used.
\ENDARGUMENTS

\RETURN

Tcl channel with reference to opened VISA session. This channel can be used in standard Tcl IO procedures, like \TCLCOMMANDREF{puts}.

\NOTES

There is no a Tcl wrapper for \VISACOMMANDREF{viClose} \VISA API function. In order to close a \VISA session one should use standard Tcl \TCLCOMMANDREF{close} command instead, which calls \VISACOMMANDREF{viClose} internally.

\EXAMPLE

\begin{verbatim} 
# open resource manager session
set rm [visa::open-default-rm]

# open instrument with default access mode and timeout
set vi1 [visa::open $rm "ASRL1::INSTR"]

# open instrument exclusively
set vi2 [visa::open $rm "ASRL2::INSTR" $visa::EXCLUSIVE_LOCK]
\end{verbatim} 

\SEEALSO

\COMMANDREF{visa::open-default-rm}

%%%%%%%%%%%%%%%%%%%%%%%%%%%%%%%%%%%%%%%%%%%%%%%%%%%%%%%%%%%%%%%%%%%%%

\COMMAND{visa::open-default-rm}

\PURPOSE

Returns a session to the Default Resource Manager resource. This command is a front-end for \VISACOMMANDREF{viOpenDefaultRM} \VISA API function.

\SYNTAX{visa::open-default-rm}

\NOARGUMENTS

\RETURN

Tcl channel with reference to opened resource manager session. This channel can be used in subsequent \tclvisa procedure calls.

\NOTES

There is no a Tcl wrapper for \VISACOMMANDREF{viClose} \VISA API function. In order to close a \VISA session one should use standard Tcl \TCLCOMMANDREF{close} command instead, which calls \VISACOMMANDREF{viClose} internally.

\EXAMPLE

\begin{verbatim} 
# open resource manager session
set rm [visa::open-default-rm]

# use session reference
...

# close session
close $rm
\end{verbatim} 

\SEEALSO

\COMMANDREF{visa::open}

\COMMAND{visa::set-attribute}

\PURPOSE

Sets the state of an attribute. This command is a front-end for \VISACOMMANDREF{viSetAttribute} \VISA API function.

\SYNTAX{visa::set-attribute session attribute attrState}

\BEGINARGUMENTS
\ARGCHANNEL
\ARGUMENT{attribute} Integer value with ID of the \VISA attribute to set. Use one of the predefned {\tt visa::ATTR\_XXX} constants.
\ARGUMENT{attrState} Integer value with desired attribute state.
\ENDARGUMENTS

\NORETURN

\EXAMPLE

\begin{verbatim} 
# open instrument with default access mode and timeout
set vi [visa::open $rm "ASRL1::INSTR"]

# set new baud rate of a serial bus
visa::set-attribute $vi $visa::ATTR_ASRL_BAUD 19200
\end{verbatim} 

\SEEALSO

\COMMANDREF{visa::get-attribute}

%%%%%%%%%%%%%%%%%%%%%%%%%%%%%%%%%%%%%%%%%%%%%%%%%%%%%%%%%%%%%%%%%%%%%

\COMMAND{visa::unlock}

\PURPOSE

Relinquishes a lock for the specified resource. This command is a front-end for \VISACOMMANDREF{viUnlock} \VISA API function.

\SYNTAX{visa::unlock session}

\BEGINARGUMENTS
\ARGCHANNEL
\ENDARGUMENTS

\NORETURN

\EXAMPLE

\begin{verbatim} 
# get exclusive lock and wait forever
visa::lock $vi

# release the lock
visa::unlock $vi
\end{verbatim} 

\SEEALSO

\COMMANDREF{visa::open}, \COMMANDREF{visa::lock}

%%%%%%%%%%%%%%%%%%%%%%%%%%%%%%%%%%%%%%%%%%%%%%%%%%%%%%%%%%%%%%%%%%%%%

%\begin{theindex}
%\input tclvisa.idx
%\end{theindex}

\end{document}
